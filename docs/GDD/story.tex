	\AddToShipoutPicture{\BackgroundPic}

\chapter{Estória}

\section*{Linha do tempo}

\textbf{Ano 1379, Reino de Arland}: Após um período de guerra civíl, Claude, líder dos rebeldes e um influente barão da região ascende ao trono.

\textbf{Ano 1386, Reino de Arland}: Nascimento da princesa de Arland, Nadine.

\textbf{Ano 1399, Reino de Arland}: Um grupo de nobres dissidentes, liderados pelo mago da corte, Richelieu, toma o reino.
	O Rei é capturado e enviado às masmorras para assistir ao reinado de Richelieu. A Rainha é assassinada. A princesa teria o mesmo destino, porém consegue escapar de seus captores e é perseguida até um penhasco a beira-mar. Encurralada, não vê nenhuma outra saída senão pular.
	Nenhum corpo é encontrado, e o novo monarca oferece um prêmio pela cabeça da princesa.

\textbf{Ano 1400, Reino de Iram}: Forçada a viver nas sombras em seu país de nascimento, a princesa foge para Iram, um reino próximo. Tendo que se esconder e roubar para viver, Nadine é resgatada por Hatem, um velho amigo de seu pai que serviu como espião durante a guerra civíl. Ele a toma como aprendiz para treiná-la para resgatar seu pai.

\textbf{Anos 1401-1406, Reino de Arland}: Rumores sobre uma ladra que tem atacado pontos estratégicos do reino libertando prisioneiros e destruindo suprimentos do exército. Opositores do rei acreditam se tratar da princesa Nadine e a apelidam de \emph{La Dauphine} devido à sua fuga milagrosa.

\textbf{Ano 1407, Reino de Arland}:  Nadine retorna a Arland para resgatar seu pai. 

\section*{Cenários}

	O jogo possuirá dois tipos de ambientes: internos e externos. Ambientes internos serão mais fechados e horizontais, enquanto os ambientes externos permitirão mais exploração vertical. A relação entre os ceńarios é mostrada na \emph{beat chart} (tabela ~\ref{bchart}.1). 
	
\section*{Progressão}

Como os ambientes do castelo são interligados, o jogador se move de um cenário para o outro a pé, o que é mostrado durante o jogo.

O jogador inicia sua jornada nas redondezas da cidade, próximo do penhasco no qual Nadine escapou anos atrás. Neste ponto o jogador será introduzido às mecânicas de plataforma do jogo (pulo, navegação utilizando o gancho, tipos de movimentação, etc.). O jogador segue a pé até chegar na vila do castelo. Este cenário já terá a presença de guardas e introduz o jogador às mecânicas de \emph{stealth} do jogo. O caminho até o portão do castelo testa as habilidades que o jogador adquiriu nos cenários anteriores. Neste ponto o jogador é forçado a tentar entrar no castelo por sua fossa.

O cenário onde a princesa deve atravessar a fossa é o primeiro estágio fechado do jogo e terá como principal mecânica maneiras de destrair os guardas para torná-los vulneráveis ou atraí-los para armadilhas. Através da fossa o jogador entra no laobratório, onde encontrará o último upgrade do jogo, as poções. Existem 3 tipos de poções: Atordoantes, de Fumaça e as poções explosivas. As poções explosivas podem ser usadas em situações de fuga ou ataque surpresa para explodir obstáculos no caminho. Os demais cenários utilizam combinações e pequenas variações das mecânicas apresentadas.


O \emph{endgame} é a parte de resgate do Rei e o confronto com o mago na sala do trono. Durante a parte de resgate será possível acontecer de o rei ser morto durante a fuga. Isto não resultará no fim do jogo, influenciando apenas no fim da estória. Entre o resgate e confronto haverá uma fase focada em exploração (Aposentos Reais), onde o jogador poderá optar por explorar o ambiente para descobrir mais sobre as motivações de Richelieu, ou seguir direto para o confronto.

	\vspace{2cm}

%	\begingroup
		\let\cleardoublepage\clearpage
		\begin{landscape}
	\hspace{-4cm}
	\begin{center}	
		\begin{table}[H]
			\label{bchart}
			\center
			\raggedbottom
		  	\begin{tabular}{| m{1.8cm} | m{1.8cm} | m{1.5cm} | m{1.5cm} | m{1.8cm}| m{1.8cm} | m{1.5cm} | m{1.5cm} | m{1.5cm} | m{1.5cm}| m{1.5cm}}
		  
		  	\hline
		  
			Localização & Redondezas da Vila  & Vila & Portões do Castelo & Fossa do Castelo & Laboratório & Biblioteca & Torre & Aposentos Reais & Sala do Trono\\ \hline
		
			Gameplay & Plataforma & \emph{Stealth} & \emph{Stealth}, Plataforma & Fuga & \emph{Stealth} & Exploração & Resgate & Exploração & \emph{Boss Fight}\\ \hline
		
			Objetivo  & Alcançar a vila & Alcançar portões do castelo & Atravessar ponte para o castelo & Escapar dos perseguidores e entrar no astelo & Obter as poções e sair do laboratório & Exploração da biblioteca para descobrir mais sobre a estória ou seguir direto para a saída & Resgatar o Rei & Explorar os aposentos reais ou seguir para a saída & Confrontar o Mago \\ \hline
		
			Avanço da Estória & Nadine alcança a vila. & Nadine chega ao portão do castelo. & Um dos guardas vê algo suspeito e Nadine é perseguida até a entrada da fossa. & Nadine evade seus perseguidores e ganha acesso ao castelo. & Nadine cria poções para utilizar na jornada. & Nadine pode descobrir mais sobre a estória. & A princesa encontra o Rei e procede para resgatá-lo. Durante o processo é possível que o rei morra. &  Nadine pode descobrir mais sobre a motivação do mago & Nadine derrota o mago e recupera seu trono. \\ \hline
		
			Nova Habilidade/Novo Item & Gancho & Ataque surpresa, Rogue Reflex	& \emph{Lockpick} & Uso de itens do cenário & Poções & - & - & - & -  \\ \hline
		
			Mecânica & Escalada & Infiltração & Fuga & Distração & Armadilhas & Infiltração & Rastreio & Infiltração & Combate \\ \hline
		
			Inimigos & - & Guarda Normal & Arqueiro, Guarda Normal & Guarda Ágil & Guarda Normal, Guarda Ágil & Guarda Pesado, Guarda Ágil & Guarda Elite & Guarda Elite & Mago \\ \hline
			
		  \end{tabular}

			\caption{\emph{Beat Chart}}
	
		\end{table}
	
	\end{center}
\end{landscape}


%	\endgroup

O confronto final será uma batalha com foco em estratégia e planejamento, onde o jogador deverá utilizar seus equipamentos e habilidades de forma consciente para vencer. Ao derrotar o mago o jogo acaba e o jogador assiste à \emph{cutscene} final. O jogo pode ter um de doís finais, dependendo da quantidade de inimigos mortos pelo jogador (Nadine pode se tornar uma rainha justa ou cruel).


